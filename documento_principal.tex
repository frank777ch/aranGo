\documentclass[12pt, a4paper, oneside]{report}

% ==============================================================================
% 1. PAQUETES BÁSICOS Y DE IDIOMA
% ==============================================================================
\usepackage[utf8]{inputenc}
\usepackage[spanish, es-tabla]{babel}
\usepackage{graphicx}
\usepackage{geometry}
\usepackage{amsmath, amssymb}
\usepackage{lipsum} % Para generar texto de relleno (borrar al final)

% ==============================================================================
% 2. CONFIGURACIÓN DE MÁRGENES Y ASPECTO
% ==============================================================================
\geometry{a4paper, left=3cm, right=2.5cm, top=3cm, bottom=3cm}
\graphicspath{{img/}} % Ruta donde LaTeX buscará las imágenes

% ==============================================================================
% 3. ENCABEZADOS Y PIES DE PÁGINA (CON FANCYHDR)
% ==============================================================================
\usepackage{fancyhdr}
\pagestyle{fancy}
\fancyhf{}
\fancyhead[L]{\nouppercase{\leftmark}}
\fancyfoot[C]{\thepage}
\renewcommand{\headrulewidth}{0.4pt}
\renewcommand{\footrulewidth}{0pt}

% ==============================================================================
% 4. BIBLIOGRAFÍA CON BIBLATEX
% ==============================================================================
\usepackage[backend=biber, style=apa, sorting=nyt]{biblatex}
\addbibresource{bib/bibliografia.bib}

% ==============================================================================
% 5. HIPERVÍNCULOS (DEBE IR CASI AL FINAL)
% ==============================================================================
\usepackage[colorlinks=true, linkcolor=black, citecolor=black, urlcolor=blue]{hyperref}

% ==============================================================================
% INICIO DEL DOCUMENTO
% ==============================================================================
\begin{document}

% --- PORTADA ---
% Este archivo contiene únicamente la portada.
\begin{titlepage}
    \centering % Centra todo el contenido de la página

    % 1. Logo de la universidad
    \includegraphics[width=0.4\textwidth]{unmsm-logo.pdf}\par % Ajusta el tamaño según necesites
    
    \vspace{1cm} % Espacio vertical
    
    % 2. Nombre de la Universidad y Facultad
    {\Large \textbf{Universidad Nacional Mayor de San Marcos}}\par
    {\large \textit{(Universidad del Perú, DECANA DE AMÉRICA)}}\par
    \vspace{0.5cm}
    {\large Facultad de Ingeniería de Sistemas e Informática}\par
    {\large Escuela Profesional de Ingeniería de Software}\par
    
    \vfill % Empuja el contenido hacia arriba y abajo, dejando espacio en el medio
    \vspace{0.5cm}
    
    % 3. Título del Informe
    {\huge \bfseries Optimización de Rutas}\par

    \vfill % Empuja el contenido hacia arriba y abajo, dejando espacio en el medio
    
    % Curso
    \begin{flushleft}
    \large \textbf{Curso:}\par
    \normalsize [Tu Nombre Completo]\par
    [Nombre del Colaborador 1]\par
    [Nombre del Colaborador 2]\par
    \end{flushleft}
    

    
    % 4. Autores y Asesor
    \begin{flushleft}
    \large \textbf{Autor(es):}\par
    \normalsize [Tu Nombre Completo]\par
    [Nombre del Colaborador 1]\par
    [Nombre del Colaborador 2]\par
    \vspace{1cm}
    \large \textbf{Asesor:}\par
    \normalsize [Nombre del Asesor]\par
    \end{flushleft}

    \vfill
    
    % 5. Lugar y Fecha
    {\large Lima, Perú}\par
    {\large \today}\par % \today inserta la fecha actual automáticamente
    
\end{titlepage}

% --- PÁGINAS PRELIMINARES CON NUMERACIÓN ROMANA ---
% \frontmatter % Comando que prepara para las páginas preliminares
\pagestyle{fancy}

\tableofcontents
\cleardoublepage

% --- Resumen ---
\chapter*{Resumen}
\addcontentsline{toc}{chapter}{Resumen}

REEMPLAZA ESTE TEXTO CON EL RESUMEN REAL DE TU PROYECTO.
El resumen debe ser conciso y presentar los puntos clave del informe: el problema, la metodología utilizada, los resultados principales y las conclusiones más importantes. Normalmente no excede las 250-300 palabras.

\vspace{1cm}
\noindent
\textbf{Palabras clave:} LaTeX, colaboración, informe, GitHub, UNMSM.
\cleardoublepage

% --- CUERPO PRINCIPAL DEL DOCUMENTO CON NUMERACIÓN ARÁBIGA ---
% \mainmatter % Comando que inicia el cuerpo principal y la numeración arábiga

\chapter{Introducción}
\label{chap:intro}

\section{Antecedentes del Problema}
\label{sec:antecedentes}
Este es el inicio de la introducción. Aquí se describe el contexto general del proyecto.
Hacer referencia a otras partes del documento es fácil, por ejemplo, la metodología se describe en el Capítulo~\ref{chap:metodos}.
La colaboración en proyectos complejos es fundamental, como lo discute Knuth en su obra sobre TeX \cite{knuth1984}.

\section{Planteamiento y Justificación}
\chapter{Metodología Aplicada}
\label{chap:metodos}

\section{Descripción del Enfoque}
\label{sec:metodologia}
En este capítulo se detalla la metodología empleada para resolver el problema.
Podemos incluir tablas, como se muestra en la Tabla~\ref{tab:ejemplo}.

\begin{table}[h!]
    \centering
    \caption{Ejemplo de una tabla descriptiva.}
    \label{tab:ejemplo}
    \begin{tabular}{|l|c|r|}
        \hline
        \textbf{Concepto} & \textbf{Símbolo} & \textbf{Valor Estimado} \\ \hline
        Velocidad de la luz & $c$ & $299,792,458 \, m/s$ \\
        Constante de Planck & $h$ & $6.626 \times 10^{-34} \, J \cdot s$ \\ \hline
    \end{tabular}
\end{table}

\section{Herramientas Utilizadas}
Las herramientas seleccionadas para este proyecto son fundamentales. Einstein sentó las bases de la física moderna \cite{einstein1905}.

% --- SECCIÓN FINAL: BIBLIOGRAFÍA ---
\cleardoublepage
\printbibliography[title={Referencias Bibliográficas}]

\end{document}