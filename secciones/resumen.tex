\chapter*{Resumen}
\addcontentsline{toc}{chapter}{Resumen}

La gestión de rutas de despacho en la última milla representa un desafío logístico crítico, especialmente en entornos urbanos complejos donde la planificación manual resulta ineficiente y costosa. Este proyecto aborda dicha problemática mediante el diseño y la propuesta de un sistema de software para la optimización de rutas, fundamentado en el Problema de Enrutamiento de Vehículos (VRP) y su conexión con el Problema del Vendedor Viajero (TSP). La solución propuesta utiliza un Algoritmo Genético como núcleo heurístico para generar rutas cuasi-óptimas que minimicen la distancia total, el tiempo de recorrido y los costos operativos asociados.

El modelo integra múltiples variables del mundo real, incluyendo la capacidad limitada de la flota, la priorización de pedidos basada en reglas de negocio específicas, y la consideración de factores dinámicos como el tráfico. El sistema propuesto se materializa en una plataforma que automatiza la asignación de rutas, pero también permite la gestión y modificación manual por parte de los administradores, así como la visualización de recorridos para los conductores. El objetivo final es ofrecer una herramienta robusta y escalable que mejore la eficiencia operativa, aumente la resiliencia logística y contribuya a la toma de decisiones estratégicas en empresas de distribución.

\vspace{1cm}
\noindent
\textbf{Palabras clave:} optimización de rutas, problema del vendedor viajero (TSP), algoritmos genéticos, logística de última milla, sistema de gestión de flotas, optimización combinatoria.